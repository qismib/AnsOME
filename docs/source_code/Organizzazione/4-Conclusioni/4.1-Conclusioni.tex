
\chapter*{Conclusioni}
% ==============================================================================================================
% Commento al lavoro svolto

Nel presente lavoro, è stata esplorata la capacità degli ansatze \inglese{quantum Unitary Coupled Cluster} ($q$-UCC) di riprodurre con precisione le energie fondamentali e di dissociazione della molecola di idruro di litio (LiH), con un focus sulla complessità circuitale richiesta da ciascuna variante. I metodi studiati, inclusi $q$-UCCS, $q$-UCCD, $q$-UCCSD, $q$-pUCCD e EfficientSU(2), hanno permesso di valutare i compromessi tra accuratezza energetica e profondità circuitale, un aspetto cruciale per una possibile implementazione su \inglese{quantum processing units} (QPU) attuali.

L’analisi ha messo in luce le peculiarità di ciascun ansatz: se da un lato $q$-UCCSD è stato il più accurato, producendo risultati molto vicini alla soglia teorica data dal modello classico Full Configuration Interaction (FCI), dall’altro richiede una profondità circuitale elevata e un numero di porte CNOT significativo, che ne limita la praticità per implementazioni hardware immediate. Al contrario, EfficientSU(2), con una struttura circuitale semplificata, ha dimostrato di poter fornire stime energetiche accettabili per determinate condizioni, nonostante richieda risorse minime. L’introduzione del \inglese{quantum} pair-UCCD ($q$-pUCCD) e delle ottimizzazioni orbitali ($q$-oo-pUCCD) ha inoltre indicato una promettente via intermedia tra la precisione e la riduzione del costo computazionale, posizionandosi come opzione interessante per simulazioni molecolari in cui le risorse quantistiche siano limitate.

Complessivamente, i risultati ottenuti offrono spunti sul potenziale degli ansatze $q$-UCC per simulazioni chimiche avanzate, sebbene l’attuale tecnologia richieda la continua ricerca di una maggiore efficienza circuitale. Questo studio potrebbe suggerire che sia possibile ottenere risultati significativi anche con architetture hardware contenute, grazie a scelte ponderate di ansatz e a ottimizzazioni.
% ==============================================================================================================
% Prospettive future
Alla luce di questi risultati, sviluppi futuri potrebbero orientarsi verso la valutazione degli ansatze in contesti più realistici. Un primo passo potrebbe essere quello di testare le configurazioni $q$-UCC attraverso simulatori quantistici con rumore, che permetterebbero di comprendere meglio l’effetto degli errori di gate e delle fluttuazioni di stato sulle stime energetiche. Questa fase sarebbe cruciale per determinare la robustezza degli ansatze rispetto agli errori di calcolo e per identificare potenziali limiti nelle implementazioni circuitali attuali.

Successivamente, e qualora i risultati su simulatori rumorosi confermassero la stabilità delle soluzioni, si potrebbe prevedere un test diretto su hardware quantistico. Tale verifica consentirebbe di validare le stime energetiche ottenute e di confrontare l’effettiva complessità circuitale con le capacità dei dispositivi fisici, aprendo la strada a simulazioni sperimentali più accurate e, eventualmente, a miglioramenti nelle strategie di ottimizzazione e nella selezione degli ansatze per applicazioni pratiche.
